\chapter{Week 1 - Lecture 1: Introduction}
\textit{5-09-2017 \\
Carolina Castaldi \\} 
\\
The first lecture covers a number of basic innovation concepts, including the distinction between invention, innovation and diffusion, the definitions of knowledge and technology, product and process innovation, radical and incremental innovation.

\section{What is the difference between innovation and invention?}
\textsc{Innovation} is the introduction of a new product into the market or the effective use of a new production process. \\
\\
\textsc{Invention} is a new idea, a scientific discovery or new technology not yet put into practice. It is mostly an individual creativity, not strictly tied to economic incentives. 

\section{What is product innovation?}
Product innovation refers to a change in the product. It can be in two different forms:
\begin{itemize}
\item First, an improvement in the performance of a product. For example, an increase in digital camera resolution. 
\item Second, new features in a product. For example, the new iPhone 7 has dual cameras which did not exist in the previous iPhones. It is also a product innovation.\\
\end{itemize}

\section{What is process innovation?}
Process innovation involves improvement in the process of producing a product. It includes changes across all the value chain activities: improved inbound logistics, better media planning, or improved manufacturing process. For example, using instant demand data to plan production run is a process improvement. It can lead to lower inventory and lower stock outs.

\section{What is incremental innovation?}
Incremental innovation is the innovation concerning a series of small improvements to an existing product or product line that usually helps maintain or improve its competitive position over tine. Incremental innovation is regularly used within the high technology business by companies that need to continue to improve their products to include new features increasingly desired by consumers. 

\section{What is radical / disruptive innovation?}
A radical or disruptive innovation is an innovation that has a significant impact on a market and on the economic activity of firms in that market. This concept focuses on the \textsc{impact} of innovations as opposed to their novelty. The innovation could, for example, change the structure of the market, create new markets or render existing products obsolete.

\section{The economics of the car industry}
Tesla changed the economics of the car industry. The car industry is highly concentrated with very high barriers to entry, mainly caused by its high Research \& Development costs (and other sunk costs) and the complexity of the processes / product. The car industry is a highly regulated industry with strong labor unions. There is also a strong reliance on suppliers for specialized parts.

\section{How is Tesla changing the economics of the car industry?}
\begin{itemize}
\item Novel approach to Research \& Development (R\&D);
\item In-house manufacturing to lower costs and decrease reliance on suppliers;
\item ICT to lower maintenance costs for users and direct retail;
\item Targeting a niche market of price-insensitive clients and going full electric.
\end{itemize}

\section{What is codified knowledge / tactic knowledge?}
\textsc{Explicit knowledge} is knowledge that can be readily articulated, codified, accessed and verbalized. It can be easily transmitted to others. Most forms of explicit knowledge can be stored in certain media. The information contained in encyclopedias and textbooks are good examples of explicit knowledge. \\
\\
\textsc{Tacit knowledge} (as opposed to formal, codified or explicit knowledge) is the kind of knowledge that is difficult to transfer to another person by means of writing it down or verbalizing it. For example, that London is in the United Kingdom is a piece of explicit knowledge that can be written down, transmitted, and understood by a recipient. However, the ability to speak a language, knead dough, play a musical instrument, or design and use complex equipment requires all sorts of knowledge that is not always known explicitly, even by expert practitioners, and which is difficult or impossible to explicitly transfer to other people.

















