\chapter{May - Overcoming Contradictions Between Growth and Sustainability: Institutional Innovation in the BRICS}
\textit{26-10-2017 \\
Jan Hidde Lavell} 

\section{Abstract}
The recent accelerated growth rates or efforts to emulate countries that have achieved a rapid pace of economic growth are widely acclaimed as means to uplift millions from poverty. In so doing, however, this rapid economic growth is most likely to coincide with unsustainable levels of consumption, place excessive pressure on life support systems and terrestrial sinks and foreshorten options for the future. Rather than pursuing the “Environmental Kuznets Curve” (EKC) hypothesis that higher income will bring with it the means to reduce the impacts of greater consumption, ecological economists assert that buying our way out of future scarcity with fast growth is indeed contradictory with sustainability. To better understand these contradictions and explore potential institutional innovations that may enable developing nations to
better confront them (in effect, “tunneling under” the EKC), this article refers to recent experience in the BRICS countries. 

\section{Introduction}
Rapid economic growth is most likely to coincide with unsustainable levels of consumption, place excessive pressure on life support systems and terrestrial sinks and foreshorten options for the future. It is a fundamental tenet of ecological economics that buying our way out of future scarcity with fast growth is in fact a recipe for disaster; such growth is intrinsically contradictory to sustainability. The aim of the article is not to compare or to emulate one or the other model, but rather to let each country’s growth path speak for itself while offering options for the rest.

\section{Conclusions}
Innovation and governance offer avenues for emerging nations to face the challenges of sustainability within the context of rapid economic change, but do not counter the underlying paradox that makes growth fundamentally unsustainable. Political will to face the difficult choices associated with resource conserving restraint is seldom available, except perhaps in the face of major natural disaster or looming man-induced catastrophe. Even when environmental problems assume a global dimension, however, innovative attempts are often stymied by hidden agendas that impede progress to reach common goals. The experience of the BRICS countries suggests that imaginative solutions may be promoted as a means to avoid “overshoot” in resource consumption. For this to occur, opportunities for South-South interchange are needed to find pathways to “tunnel under” the EKC. This will require, in turn, the help of propitious international terms of trade and institutional arrangements, costless and smooth technology transfers and above all, societies willing to forego current consumption for future social benefit and environmental quality.  

