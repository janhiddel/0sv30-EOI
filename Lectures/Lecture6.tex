\chapter{Week 6 - UN-Agenda 21: Actors and actions outside the UN framework}
\textit{09-10-2017 \\
Henny Romijn}

\section{What is Agenda 21?}
Agenda 21 is the sustainable development action plan adopted at the Rio Summit in 1992, containing 2500 action points. \\
\\
Agenda 21 has some heavy emphasis on procedures with increased popular participation and new forms of democracy. There is a special role for local authorities, but within the framework of a national campaign managed by national government. \\
\\
Limitations and criticisms: Agenda 21 is voluntary, non-binding and has varying commitment by national governments. It put constraints on local resources, or on their mobilisation. Also, many problems are too big for local governments.

\section{What is LA21?}
LA21, abbreviation for Local Agenda 21: the implementation of Agenda 21 was intended to involve action at international, national, regional and local levels. Some national and state governments have legislated or advised that local authorities take steps to implement the plan locally These programs are often known as "Local Agenda 21" or "LA21". \\
\\
Limitations and criticisms: according to some, LA21 is a stealthy imposition of a new world government curtailing our liberties and life styles, our rights as individuals. Also, the bureaucratic process behind it is widely criticized.

\section{What is Baker's assesment on LA21?}
\textsc{Advantage}: it serves the public awareness raising, involvement, education, deeper local democracy and networking. \\
\\
\textsc{Disadvantage}: Effectiveness of local SD processes constrained by governance at higher levels (national, regional, global).

\section{What is ICLEI?}
ICLEI, abbreviation for International Council for Local Environmental Authorities, was founded in 1990 and is an international association of local governments and national and regional local government organizations that have made a commitment to sustainable development. ICLEI formulated guidelines, good implementation principles, timetables and targets. The organization is part of a long term vision and plan for sustainable development. \\
\\
However, many LA21 Local Action Plans do not meet all ICLEI criteria. Moreover ICLEI has just over 1500 cities, towns and regions as members, and 80\% of these are in Europe.

\section{What are the radical pro-SD countermovements outside the UN framework?}
\begin{itemize}
\item Non-governmental environmental organizations (like Greenpeace);
\item World Social Forum: "Another world is possible";
\item ATTAC, Association pour la Taxation des Transactions pour l'Aide aux Citoyens: "The world is not for sale";
\item Via Campesina: Coalition of more than 148 organizations representing the interests of sustainable and organic family farming across 69 countries 
\end{itemize} 

Major point of criticism by the countermovements: the lack of attention to social issues (as compared to environmental issues) in the UN process

\section{What are iconic individuals?}
\begin{itemize}
\item Rigoberta Menchú, Nobel Peace Prize winner, from an indigenous Indian community in Guatemalea
\item Ela Bhatt, Indian social reformer, founder of the Indian Self Employed Women Association (SEWA)
\item Wangari Maathai, Nobel Peace Prize winner from rural Kenya, founder of a global tree planting movement – the Green Belt Movement – that has already planted 51 million trees.
\end{itemize}

\section{What is social entrepreneurship?}
Social entrepreneurship is the use of the techniques by start-up companies and other entrepreneurs to develop, fund and implement solutions to social, cultural, or environmental issues. Social entrepreneurs design innovative business models for poverty reduction and sustainable development. They negotiate the trade-offs between People, Planet, Profit goals, focusing on Win-Win-Win
solutions.

\clearpage