\chapter{Van den Bergh - A critism of "degrowth" and a plea for "a-growth"}
\textit{26-10-2017 \\
Jan Hidde Lavell} 

\section*{Abstract}
In recent debates on environmental problems and policies, the strategy of “degrowth” has appeared as an alternative to the paradigm of economic growth. This new notion is critically evaluated by considering five common interpretations of it. One conclusion is that these multiple interpretations make it an ambiguous and rather confusing concept. Another is that degrowth may not be an effective, let alone an efficient strategy to reduce environmental pressure. It is subsequently argued that “a-growth,” i.e. being indifferent about growth, is a more logical social aim to substitute for the current goal of economic growth, given that GDP (per capita) is a very imperfect indicator of social welfare. In addition, focusing ex ante on public policy is considered to be a strategy which ultimately is more likely to obtain the necessary democratic–political support than an ex ante, explicit degrowth strategy. In line with this, a policy package is proposed which consists of six elements, some of which relate to concerns raised by degrowth supporters.

\section*{Introduction}
Van den Bergh uses three criteria: 
\begin{itemize}
\item Does degrowth reduce environmental pressure?
\item Is degrowth likely to receive social and democratic-political support?
\item Is degrowthe conomically efficient/cost-effective?
\end{itemize}

A new notion, namely “degrowth,” has been suggested as a possible
alternative to the paradigm of economic growth. This article aims to evaluate degrowth interpretations and strategies from two main angles, namely environmental effectiveness and social–political feasibility. The experience of Van den Bergh in discussing and
reading about degrowth is that it is defined and interpreted in multiple ways. Van den Bergh has come across five main interpretations of
degrowth:
\begin{enumerate}
\item \textsc{GDP degrowth}: negative GDP growth or a reduction in GDP (Gross Domestic Product);
\item \textsc{Consumption degrowth}: the second interpretation of degrowth means striving for a reduction in the amount of consumption. Such a strategy is then hoped to translate into less resource use and less pollution; 
\item \textsc{Work-time degrowth}: triving for degrowth in
terms of less work hours (a shorter working week or year);
\item \textsc{Radical degrowth}: perhaps for the majority of degrowth proponents the notion of degrowth denotes a radical change of (or many radical changes in) the economy. This may involve changes in values, ethics, preferences, financial systems, markets (versus informal exchange), work and labor, the role of money, or even profit-making and ownership; 
\item \textsc{Physical degrowth}: can be defined as a reduction of the physical size of the economy, notably in terms of resource use and polluting emission.
\end{enumerate}

\section*{Main critiques}

\begin{enumerate}

\item \textsc{GDP degrowth}
\begin{itemize}
\item Short and long terms effect uncertain;
\item Reversing causalities: first policies, then (maybe) consequence of degrowth;
\item Restort to dirtier technologies: greentech's should grown, dirty tech's not.
\end{itemize}

\item \textsc{Consumption degrowth}
\begin{itemize}
\item Measurement-indicator problems;
\item Which consumption limit?
\item Rebound mechanism (decrease in one, increase in another good, savings here-more money to spend there).
\end{itemize}

\item \textsc{Work-time degrowth}
\begin{itemize}
\item Attractive, but does not guarantee more reduction of “dirty” than of “cleaner” consumption,
\end{itemize}

\item \textsc{Radical degrowth}
\begin{itemize}
\item Grand, imprecise idea, impossible to gain broad political support
\end{itemize}

\item \textsc{Physical degrowth}
\begin{itemize}
\item Old wine in new bottles: around since the 60s; has no specific policy angle;
\item Ineffective: scale down to 50\% 
\end{itemize}

\end{enumerate}

\section*{Agrowth instead of degrowth}
Agrowth is being neutral or indifferent to economic growth. Critique on GDP:
\begin{itemize}
\item GDP is not a good indicator of social wellbeing;
\item Beyond a threshold income level the cost of growth exceeds its benefits;
\item GDP’s focus on market transactions excludes informal transactions between people;
\item GDP does not account for “goods and services” delivered by nature.
\end{itemize}

\section*{An effective policy package}
\begin{enumerate}
\item Effective international agreement;
\item Encourage different work-time norms;
\item Regulate commercial advertisement (especially on status goods)
\item Communicate and inform to encourage behavioral changes;
\item Ignore/give less importance to GDP;
\item Technology-specific policies (e.g. to direct research)
\end{enumerate}

\section*{Conclusion}
The concept of degrowth is ambiguous since van den Bergh identified 5 different interpretations. The notion of degrowth is not useful since it is not effective in reducing pressure on environment and not politically feasible. The concept of agrowth is less ambiguous/better. 

\clearpage
