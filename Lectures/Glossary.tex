\chapter{Glossary}

\titleformat*{\section}{\large\bfseries}

\textit{1-11-2017 \\
Jan Hidde Lavell} \\
\\
This is a list of most important concepts that have been discussed during the 0SV30 course. The list is
not encompassing all possible terms but is merely an indication.

\begin{sortEnvironment}{Economies of scale}
Economies of scale is the cost advantage that arises with increased output of a product. Economies of scale arise because of the inverse relationship between the quantity produced and per-unit fixed costs; i.e. the greater the quantity of a good produced, the lower the per-unit fixed cost because these costs are spread out over a larger number of goods.
\end{sortEnvironment}

\begin{sortEnvironment}{Innovation}
Innovation is the introduction of a new product into the market or the effective use of a new production process. 
\end{sortEnvironment}

\begin{sortEnvironment}{Invention}
Invention is a new idea, a scientific discovery or new technology not yet put into practice. It is mostly an individual creativity, not strictly tied to economic incentives.
\end{sortEnvironment}

\begin{sortEnvironment}{Process innovation}
Process innovation involves improvement in the process of producing a product. It includes changes across all the value chain activities: improved inbound logistics, better media planning, or improved manufacturing process. 
\end{sortEnvironment}

\begin{sortEnvironment}{Incremental innovation}
Incremental innovation is the innovation concerning a series of small improvements to an existing product or product line that usually helps maintain or improve its competitive position over tine.
\end{sortEnvironment}

\begin{sortEnvironment}{Radical / disruptive innovation}
A radical or disruptive innovation is an innovation that has a significant impact on a market and on the economic activity of firms in that market and focuses on the impact of innovations as opposed to their novelty. 
\end{sortEnvironment}

\begin{sortEnvironment}{Codified knowledge}
Codified / explicit knowledge is knowledge that can be readily articulated, codified, accessed and verbalized. It can be easily transmitted to others. Most forms of explicit knowledge can be stored in certain media. 
\end{sortEnvironment}

\begin{sortEnvironment}{Tacit knowledge} 
Tactic knowledge is the kind of knowledge that is difficult to transfer to another person by means of writing it down or verbalizing it.
\end{sortEnvironment}

\begin{sortEnvironment}{Diseconomies of scale}
Diseconomies of scale are the cost disadvantages that firms and governments accrue due to increase in firm size or output, resulting in production of goods and services at increased per-unit costs. This typically follows the law of diminishing returns, where further increase in size of output will result in even greater increase in average cost.
\end{sortEnvironment}

\begin{sortEnvironment}{Experience Learning Curve}
The experience learning curve is a concept developed by the Boston Consulting Group in the 1960's. It is representative of a consistent relationship between the cost of production and the cumulative production quantity (total quantity produced from the first unit to the last). The experience curve implies that the more experience a firm has in producing a particular product, the lower are its costs.
\end{sortEnvironment}