\chapter{Week 7 - Concluding lecture: Question and answer}
\textit{16-10-2017 \\
Johanna Höffken}

\section{What are the steps taken for the promotion of sustainable development?}

\begin{itemize}
\item The promotion of sustainable development began with environmental critiques of the conventional model of development;
\item After this, it was no longer possible to see development isolated from its environmental and social consequences;
\item The conventional model was also criticized by its overreliance on markets for service and good distribution;
\item The pace and form of the development of the West cannot be pursued without harming the environment and its resources.
\end{itemize}

\section{How was sustainable development built upon a new SD theme?}

Sustainable development was build upon environmental, social and economic considerations. Besides this, it was build on normative principles. The promotion of equity in access to the planet's limited resources became important: the principle of equity extends across space, time and different generations. 

\section{What was the role of the UN in SD?}

The United Nations shapes understanding of and engagement with sustainable development promotion. It held various summits, where especially the Rio Summit is important:
\begin{itemize}
\item For the normative and governance principles;
\item For the building of environmental institution
\item For the engagement of the civil society, in particular the LA21; 
\item For its practical considerations of implementation (the WSSD is an example of this with monitoring and implementating).
\end{itemize}

\section{How is UN's role in sustainable development critiqued?}
The United Nations helped to structure legal, institutional, political and economic agreements of respect to sustainable development, from international down to local level. However:
\begin{itemize}
\item There should be more focus on institutional building than on policy implementation;
\item The political will among the UN member state is lacking, especially in terms of economical commitments;
\item The question arises if the UN really able to address underlying causes of unsustainable forms of development;
\item The UN is mainly about the distribution of power and priority setting (economical and social);
\end{itemize}

\section{How did ecological modernization promote weak sustainable development?}
The EU shifted its understanding of nature and the environmental problem. It was no longer seen as merely an issue of pollution control. Instead, attention has shifted from concerns with resource management and ensuring that economic development resulted in improved quality of life (decoupling) to an elaborate new form of environmental government. Still, it is mostly about declaratory commitments. 

\clearpage