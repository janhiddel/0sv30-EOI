\chapter{Kumi et al - Can post-2015 sustainable development goals survive
neoliberalism?}
\textit{26-10-2017 \\
Jan Hidde Lavell} 

\section*{Abstract}
The Rio+20 Summit of the United Nations in Brazil in 2012 committed governments to formulate a set of sustainable development goals (SDGs) that would be integrated into the Millennium Development Goals (MDGs) following its expiration in 2015. This decision has pushed sustainable development agenda into the limelight of development once again. Meanwhile, we note that the development agenda of many developing countries has been dominated by neoliberal orientation driven by market reforms, social inequality, and a move towards enhancing the economic competitiveness of the supply side of the economy. In this paper, we discuss the relationship between neoliberal economic agenda and sustainable development. We do so by examining how neoliberal policies of privatization, trade liberalization and reduction in governments spending stand to affect the attainment of sustainable development ideals and their implications on the post-2015 Sustainable Development Goals. The paper then suggests that relying solely on the mechanisms of the market in governing and allocating environmental resources is necessarily insufficient and problematic and therefore calls for a new approach—one which goes beyond just recognizing the interdependency among social, environmental and economic goals and places issues of equity and addressing unfavorable power relations at the center of interventions aimed at achieving the ideals of sustainable development.

\section*{Conclusions}
This paper draws together the discussion on neoliberal economic agenda and how it stands to affect progress towards sustainable development. It addresses the issue of the market and sustainable development. In particular, the paper has shown that the tenets of neoliberal economic agenda such as commodification, deregulation, privatization and cuts in government expenditure may in some context undermine the attainment of sustainable development by increasing poverty and inequality. This in turn increases the exploitation of environmental resources such as forests as a result of poverty-induced constraints. Additionally, the regulatory capacity of environmental management provided by the state has been reduced mainly due to budgetary constraints imposed by the adoption of neoliberalism. The effects of neoliberal policy preferences and liberalization on sustainable development are of mixed reactions; however, market-led policies provide incentives for the operations of transnational corporations which in turn may have consequential effects on the environment and social equity goals. \\
\\
This paper concludes that progress made in advancing sustainable development as an ideal goal of development over the past years remains to be threatened by the rise and expansion of neoliberal regimes in developing countries. We are therefore of the view that the economic thinking on neoliberalism will have implications on the ongoing sustainable development goals being prepared to succeed the Millennium Development Goals after 2015. This paper therefore suggests that relying solely on the mechanisms of the market in governing and allocating environmental resources is necessarily insufficient and problematic and therefore calls for a new approach—one which goes beyond just recognizing the interdependency among social, environmental and economic goals and places issues of equity and addressing unfavorable power relations at the center of interventions aimed at achieving the ideals of sustainable development. 
\clearpage