\chapter{Glossary}

\titleformat*{\section}{\large\bfseries}

\textit{19-10-2017 \\
Jan Hidde Lavell} \\
\\
This is a list of most important concepts that have been discussed during the 0SV00 course. The list is
not encompassing all possible terms but is merely an indication. To pass exam every student should
know what these concepts mean and how to use them.

\begin{sortEnvironment}{SOOD}
The Scientific Outlook on Development (SOOD) rebukes growth-at-all-cost economic strategy unleashed by Mao Zedong and Deng Xiaoping. 
\end{sortEnvironment}

\begin{sortEnvironment}{Bruntland definition of sustainable development}
Development that meets the needs of the present without compromising the ability of future generations to meet their own needs.
\end{sortEnvironment}

\begin{sortEnvironment}{Intra-generational equity}
Equity within own generation
\end{sortEnvironment}

\begin{sortEnvironment}{Inter-generational equity}
Equity between generations, that is, including the needs of future generations in the design and implementation of current policies
\end{sortEnvironment}

\begin{sortEnvironment}{Neoliberalism}
Neoliberalism is a policy model of social studies and economics that transfers control of economic factors to the private sector from the public sector associated with a laissez-faire attitude, favouring free trade, privatization, minimal government intervention in business and more. 
\end{sortEnvironment}

\begin{sortEnvironment}{Common but differentiated responsibility principle}
The common but differentiated responsibilities principle establishes that all states are responsible for addressing global environmental destruction yet they are not equally responsible. The principle balances, on the one hand, the need for all states to take responsibility for global environmental problems and, on the other hand, the need to recognize the wide differences in levels of economic development between states. These differences in turn are linked to the states’ contributions to the problems as well as their abilities to address these problems. 
\end{sortEnvironment}

\begin{sortEnvironment}{Ecological modernization}
Ecological modernization is an optimistic school of thought in the social sciences that argues that the economy benefits from moves towards environmentalism. It is an analytical approach as well as a policy strategy. 
\end{sortEnvironment}

\begin{sortEnvironment}{Third World}
"Third World" is an old term from the Cold War era during the 70s. A better concept for the 21st century is the global south.
\end{sortEnvironment}

\begin{sortEnvironment}{Agenda 21}
Agenda 21 is the sustainable development action plan adopted at the Rio Summit in 1992, containing 2500 action points.
\end{sortEnvironment}

\begin{sortEnvironment}{LA21}
LA21, abbreviation for Local Agenda 21: the implementation of Agenda 21 was intended to involve action at international, national, regional and local levels. Some national and state governments have legislated or advised that local authorities take steps to implement the plan locally These programs are often known as "Local Agenda 21" or "LA21".
\end{sortEnvironment}

\begin{sortEnvironment}{ICLEI}
ICLEI, abbreviation for International Council for Local Environmental Authorities, was founded in 1990 and is an international association of local governments and national and regional local government organizations that have made a commitment to sustainable development.
\end{sortEnvironment}

\begin{sortEnvironment}{Social entrepreneurship}
Social entrepreneurship is the use of the techniques by start-up companies and other entrepreneurs to develop, fund and implement solutions to social, cultural, or environmental issues.
\end{sortEnvironment}

\begin{sortEnvironment}{Capitalism / Communism}
Capitalism is an economic system in which capital goods are owned by private individuals or businesses. The production of goods and services is based on supply and demand in the general market (market economy), rather than through central planning (planned economy or command economy). \\
\\
Communism is a political and economic ideology based on the holding of all property in common, actual ownership being ascribed to the community as a whole or to the state.

\end{sortEnvironment}

\begin{sortEnvironment}{Decoupling}
Decoupling is reducing the amount of physical emissions or natural resource use per unit of economic output. This can be achieved either trough increased efficiency, 'eco-efficiency', stemming from technological change, or a shift to less environmentally damaging products. 
\end{sortEnvironment}

\begin{sortEnvironment}{Precautionary principle}
The Precautionary Principle states that where there are threats of serious or irreversible damage, lack of full scientific certainty shall not be used as a reason for postponing cost-effective measures to prevent environmental degradation.
\end{sortEnvironment}

\begin{sortEnvironment}{Polluter pays principle}
The 'polluters pays' principle is the commonly accepted practice that those who produce pollution should bear the costs of managing it to prevent damage to human health or the environment. For instance, a factory that produces a potentially poisonous substance as a byproduct of its activities is usually held responsible for its safe disposal. 
\end{sortEnvironment}

\begin{sortEnvironment}{Green economics}
Green economics is a methodology of economics that supports the harmonious interaction between humans and nature and attempts to meet the needs of both simultaneously
\end{sortEnvironment}

\begin{sortEnvironment}{Environmental Kuznets Curve}
The environmental Kuznets curve suggests that economic development initially leads to a deterioration in the environment, but after a certain level of economic growth, a society begins to improve its relationship with the environment and levels of environmental degradation reduces.
\end{sortEnvironment}

\begin{sortEnvironment}{De-growth / A-growth}
\textsc{De-growth} is a political, economic, and social movement based on ecological economics, anti-consumerist and anti-capitalist ideas. It is also considered an essential economic strategy responding to the limits-to-growth dilemma. Degrowth thinkers and activists advocate for downscaling of production and consumption. \\
\\
\textsc{A-growth} is being neutral or indifferent to economic growth. 
\end{sortEnvironment}

\begin{sortEnvironment}{BRICS}
The emerging economies Brazil, Russia, India, Costa Rica, Sout-Africa: they made a transition from a developing to a developed country
\end{sortEnvironment}

\begin{sortEnvironment}{Prosperity}
Synonym for welfare: a successful, flourishing, or thriving condition, especially in financial terms.
\end{sortEnvironment}

\begin{sortEnvironment}{Commodification}
Commodification is the transformation of goods, services, ideas and people into commodities, or objects of trade.
\end{sortEnvironment}

\begin{sortEnvironment}{Privatization}
Privatization refers to the act of transferring ownership of specified property or business operations from a government organization to a privately owned entity
\end{sortEnvironment}

\begin{sortEnvironment}{Deregulation}
Deregulation is the reduction or elimination of government power in a particular industry, usually enacted to create more competition within the industry. 
\end{sortEnvironment}

\begin{sortEnvironment}{Convergence} 
Conference refers to mechanisms and pathways that lead towards sustainability with a specific focus on 'Equity within biological planetary limits'. These pathways and mechanisms explicitly advocate equity and recognize the need for redistribution of the Earth's resources in order for human society to operate enduringly within the Earth's biophysical limits. 
\end{sortEnvironment}

\begin{sortEnvironment}{Baker's ladder of sustainable development}
\textsc{Pollution control:} environmental protection is important, but should not put limits on development or constrain human freedom to innovate. Technology can solve any environmental problem. Provides Environmental Kuznets Curve as evidence.\\
\\
\textsc{Weak sustainable development:} economic approach to sustainable development. Pricing is important for natural resources: best way to preserve them. Strong belief in technological substitutes. Ecological modernization: economy benefits from environment. Policies promoting economic growth. \\
\\
\textsc{Strong sustainable development:} environment a pre-condition for development, not the other way around. Limits to technological fixes. Shift from growth to non-material aspects of development. Attention to development in other world regions. From quantitative GDP to qualitative HDI and quality of life. \\
\\
\textsc{Ideal sustainable development:} attributes equal value to all life forms. Non-interference with nature. Rejects managerial interference in nature. Laborintensive technologies. Bottom up community structures 
\end{sortEnvironment}

\begin{sortEnvironment}{Environmental leapfrogging}
Environmental leapfrogging is the skipping of pollution intensive stages of development. Consequently, vintage technologies will be leapfrogged bu avoiding investments in older technologies.
\end{sortEnvironment}

\begin{sortEnvironment}{Paris agreement}
The Paris Agreement is an agreement within the United Nations Framework Convention on Climate Change (UNFCCC) dealing with greenhouse gas emissions mitigation, adaptation and finance. The agreement was negotiated by representatives of 196 parties at the 21st Conference of the Parties in Paris and adopted by consensus in 2015. In the Paris Agreement, each country determines, plans and regularly reports its own contribution it should make in order to mitigate global warming. There is no mechanism to force a country to set a specific target by a specific date.
\end{sortEnvironment}

\begin{sortEnvironment}{Overconsumption}
Overconsumption is a situation where resource use has outpaced the sustainable capacity of the ecosystem
\end{sortEnvironment}

\begin{sortEnvironment}{SEA}
Scientists and Engineers for America (SEA) was an organization focused on promoting sound science in American government, and supporting candidates who understand science and its applications.
\end{sortEnvironment}

\begin{sortEnvironment}{EAP}
Environmental Action Programme of the European Union. EAPs have guided the development of EU environmental policy since the early 1970s. 
\end{sortEnvironment}

\begin{sortEnvironment}{EPIs}
Environmental Policy Integration. Is used in the literature to capture the need to enhance coherence between sectoral, economic and environmental policies and between them and sustainable development policies.
\end{sortEnvironment}

\begin{sortEnvironment}{Transition economy}
A transition economy is an economy which is changing from a centrally planned economy to a market economy.
\end{sortEnvironment}

\begin{sortEnvironment}{Catch-up idea}
The idea that the Third World should ‘catch up’ with the developed world. Important for the conventional development model: necessary for Third World societies to ‘catch up’ with the Western style of development. This means opening up their economies to Western values,
influences, investment and trade, thereby becoming more integrated into the global market system.
\end{sortEnvironment}

\begin{sortEnvironment}{Rostow's model of development}
In the conventional model, society is understood to go through different ‘stages of economic growth’ (Rostow, 1960). Traditional societies develop to a stage of economic ‘take-off’. This sees new industries and entrepreneurial classes emerge. In ‘maturity’, steady economic growth outstrips population growth. A ‘final stage’ is reached when high mass consumption allows the emergence of social welfare. This model of development assumes a linear progression and is highly criticized.  
\end{sortEnvironment}

\begin{sortEnvironment}{Grassroots movement}
A grassroots movement (often referenced in the context of a political movement) is one which uses the people in a given district as the basis for a political or economic movement. Grassroots movements and organizations use collective action from the local level to effect change. Grassroots movements are associated with bottom-up and are sometimes considered more natural than traditional power structures.
\end{sortEnvironment}

\begin{sortEnvironment}{Bottom-up / top-down}
\textsc{A bottom-up} approach to changes one that works from the grassroots—from a large number of people working together, causing a decision to arise from their joint involvement. A decision by a number of activists, students, or victims of some incident to take action is a "bottom-up" decision. A bottom-up approach can be thought of as "an incremental change approach that represents an emergent process cultivated and upheld primarily by frontline workers" 
\textsc{A top-down} approach is where an executive decision maker or other top person makes the decisions of how something should be done. This approach is disseminated under their authority to lower levels in the hierarchy, who are, to a greater or lesser extent, bound by them.
\end{sortEnvironment}

\begin{sortEnvironment}{Rio Earth Summit}
The Rio Earth Summit was a major United Nations conference held in Rio de Janeiro in 1992. In 2012, the United Nations Conference on Sustainable Development was also held in Rio, and is also commonly called Rio+20. An important achievement of the summit was an agreement on the Climate Change Convention which in turn led to the Kyoto Protocol and the Paris Agreement
\end{sortEnvironment}

\begin{sortEnvironment}{MDGs}
The Millennium Development Goals (MDGs) were the eight international development goals for the year 2015 that had been established following the Millennium Summit of the United Nations in 2000, following the adoption of the United Nations Millennium Declaration. All 191 United Nations member states at that time, and at least 22 international organizations, committed to help achieve the following Millennium Development Goals by 2015: \\
\\
1. To eradicate \textsc{extreme poverty and hunger} \\
2. To achieve universal \textsc{primary education} \\
3. To promote \textsc{gender equality} and \textsc{empower women} \\
4. To reduce \textsc{child mortality} \\
5. To improve \textsc{maternal health} \\
6. To combat \textsc{HIV/AIDS, malaria, and other diseases} \\
7. To ensure \textsc{environmental sustainability} \\
8. To develop a \textsc{global partnership} for development
\end{sortEnvironment}

\begin{sortEnvironment}{SDGs}
The Sustainable Development Goals (SDGs) is a set of 17 goals with 169 targets among them. The Resolution is a broader intergovernmental agreement that acts as the Post 2015 Development Agenda (successor to the Millennium Development Goals). It is a non-binding document released as a result of Rio+20 Conference held in 2012 in Rio de Janeiro. \\
\\
\textsc{
1. No poverty \\
2. No hunger \\
3. Good health \\
4. Quality education \\
5. Gender equality \\
6. Clean water and sanitation \\
7. Renewable energy \\
8. Good jobs and economic growth \\
9. Innovation and infrastructure \\
10. Reduced inequalities \\
11. Sustainable cities and communities \\
12. Responsible consumption \\
13. Climate action \\
14. Life below water
15. Life on land
16. Peace and justice \\
17. Partnership for the goals
} 

\end{sortEnvironment}

\begin{sortEnvironment}{Cleantech}
Cleantech is a shortened form of clean technologies, a term used to describe an investment philosophy used by investors seeking to profit from environmentally friendly companies. Cleantech firms seek to increase performance, productivity and efficiency by minimizing negative effects the environment.
\end{sortEnvironment}

\begin{sortEnvironment}{Techno-fix}
A technological fix refers to the attempt of using engineering or technology to solve a problem, often created by earlier technological interventions. 
\end{sortEnvironment}

\begin{sortEnvironment}{End-of-pipe solutions}
Methods used to remove already formed contaminants from a stream of air, water, waste, product or similar. These techniques are called 'end-of-pipe' as they are normally implemented as a last stage of a process before the stream is disposed of or delivered. 
\end{sortEnvironment}

\begin{sortEnvironment}{Tipping point (climatology)}
A climate tipping point is a somewhat ill-defined concept of a point when global climate changes from one stable state to another stable state, in a similar manner to a wine glass tipping over. After the tipping point has been passed, a transition to a new state occurs. The tipping event may be irreversible, comparable to wine spilling from the glass: standing up the glass will not put the wine back
\end{sortEnvironment}

\begin{sortEnvironment}{Safe operating space}
A framework based on 'planetary boundaries'. These boundaries define the safe operating space for humanity with respect to the Earth system and are associated with the planet's biophysical subsystems or processes. 
\end{sortEnvironment}

\begin{sortEnvironment}{LDCs}
The Least Developed Countries (LDCs) is a list of the countries that, according to the United Nations, exhibit the lowest indicators of socioeconomic development, with the lowest Human Development Index ratings of all countries in the world. 
\end{sortEnvironment}

\begin{sortEnvironment}{Trickle-down}
An economic system in which the poorest gradually benefit as a
result of the increasing wealth of the richest. Widely criticized. 
\end{sortEnvironment}

\begin{sortEnvironment}{Impoverishment}
The process of becoming poor; loss of wealth.
\end{sortEnvironment}

\begin{sortEnvironment}{Imperialism}
Imperialism is an action that involves a country (usually an empire or a kingdom) extending its power by the acquisition of territories. It may also include the exploitation of these territories, an action that is linked to colonialism. Colonialism is generally regarded as an expression of imperialism
\end{sortEnvironment}

\begin{sortEnvironment}{Neocolonism}
Neocolonialism or neo-imperialism is the practice of using capitalism, globalization and cultural imperialism to influence a developing country in lieu of direct military control or indirect political control. 
\end{sortEnvironment}

\begin{sortEnvironment}{G77}
The Group of 77 (G77) at the United Nations is a coalition of developing nations, designed to promote its members' collective economic interests and create an enhanced joint negotiating capacity in the United Nations. There were 77 founding members of the organization, but by November 2013 the organization had since expanded to 134 member countries. Since China participates in the G77 but does not consider itself to be a member, all official statements are issued in the name of The Group of 77 and China.
\end{sortEnvironment}

\begin{sortEnvironment}{TNC}
The Nature Conservancy is a global charitable environmental organization with more than 1 million members. The TNC acts under the slogan: "Protecting nature. Preserving life". 
\end{sortEnvironment}

\begin{sortEnvironment}{CONGOs}
Government Organized Non-Governmental Organization: an independent, international, non-profit membership association of non-governmental organizations (NGOs). It facilitates the participation of NGOs in United Nations debates and decision-making.
\end{sortEnvironment}

\begin{sortEnvironment}{NGOs}
Non-governmental organizations are international organizations and generally nonprofit organizations independent of specific governments (though often funded by governments) that are active in humanitarian, educational, healthcare, public policy, social, human rights, environmental, and other areas to effect changes according to their objectives. They are thus a subgroup of all organizations founded by citizens, which include clubs and other associations that provide services, benefits, and premises only to members.
\end{sortEnvironment}

\begin{sortEnvironment}{Limits to growth}
The Limits to Growth is a 1972 book about the computer simulation of exponential economic and population growth with finite resource supplies. The book used a model to simulate the consequence of interactions between the Earth's and human systems. The original version presented a model based on five variables: world population, industrialization, pollution, food production and resources depletion. These variables are considered to grow exponentially, while the ability of technology to increase resources is only linear. The purpose of The Limits to Growth was not to make specific predictions, but to explore how exponential growth interacts with finite resources. Because the size of resources is not known, only the general behavior can be explored.
\end{sortEnvironment}

\begin{sortEnvironment}{Malthus}
Thomas Malthus (1766 -1834) was a political economist and Enlightenment thinker who observed the growing population with increasing concern. To explain poverty, dearth and famine he wrote a famous essay at the end of the 18th century entitled An Essay on the Principle of Population. 
\end{sortEnvironment}

\begin{sortEnvironment}{Non-aligned movement}
The Non-Aligned Movement is a group of states that are not formally aligned with or against any major power bloc. As of 2012, the movement has 120 members. The purpose of the organization has been enumerated as to ensure "the national independence, sovereignty, territorial integrity and security of non-aligned countries" in their struggle against imperialism, (neo)colonialism and domination. Membership is particularly concentrated in countries considered to be developing or part of the Third World, though the Non-Aligned Movement also has a number of developed nations such as Chile and Saudi Arabia.
\end{sortEnvironment}

\begin{sortEnvironment}{Anthropocentric / ecocentric}
\textsc{Anthropocentric} is the philosophy that the wealth of nature is seen only in relation to what it can provide in the service of humankind. \\
\\
\textsc{Ecocentric} is the philosophy that nature has an intrinsic value.
\end{sortEnvironment}

\begin{sortEnvironment}{Emission trading}
Emission trading, as set out in the Kyoto Protocol, allows countries that have emission units to spare (emissions permitted but not used) to sell this excess capacity to countries that are over their targets.
\end{sortEnvironment}

\begin{sortEnvironment}{Disruptive innovation}
Disruptive innovation is a term in the field of business which refers to an innovation that creates a new market and value network and eventually disrupts an existing market and value network, displacing established market leading firms, products, and alliances. 
\end{sortEnvironment}

\begin{sortEnvironment}{Tied aid}
Tied aid is an aid that is given under the condition that part or all of it must be used to purchase goods from the country providing the aid. From this it follows that untied aid has no geographical limitations.
\end{sortEnvironment}

\begin{sortEnvironment}{Good governance}
Good governance is about the processes for making and implementing decisions. It’s not about making ‘correct’ decisions, but about the best possible process for making those decisions, that is, transparant, accountable, according to the rules of law, equitable and participatory. 
\end{sortEnvironment}

\begin{sortEnvironment}{Carrying capacity}
The carrying capacity of a biological species in an environment is the maximum population size of the species that the environment can sustain indefinitely, given the food, habitat, water, and other necessities available in the environment. 
\end{sortEnvironment}

\begin{sortEnvironment}{BOP}
BOP, the bottom of the pyramid, bottom of the wealth pyramid or the bottom of the income pyramid, is the largest, but poorest socio-economic group.
\end{sortEnvironment}

\begin{sortEnvironment}{IPCC}
The Intergovernmental Panel on Climate Change (IPCC) is a scientific and intergovernmental body under the the United Nations, set up at the request of member governments, dedicated to the task of providing the world with an objective, scientific view of climate change and its political and economic impacts. The ultimate objective of the UNFCCC is to stabilize greenhouse gas concentrations in the atmosphere at a level that would prevent dangerous anthropogenic (i.e., human-induced) interference with the climate system.
\end{sortEnvironment}

\begin{sortEnvironment}{UNFCCC / COP}
\textsc{The United Nations Climate Change Conferences (UNFCC)} are yearly conferences held in the framework of the United Nations Framework Convention on Climate Change (UNFCCC). They serve as the formal meeting of the UNFCCC Parties (Conference of the Parties, COP) to assess progress in dealing with climate change, and beginning in the mid-1990s, to negotiate the Kyoto Protocol to establish legally binding obligations for developed countries to reduce their greenhouse gas emissions. \\
\\
\textsc{COP} is the supreme decision-making body of the Convention. All States that are Parties to the Convention are represented at the COP, at which they review the implementation of the Convention and any other legal instruments that the COP adopts and take decisions necessary to promote the effective implementation of the Convention, including institutional and administrative arrangements.
\end{sortEnvironment}

\begin{sortEnvironment}{Deep ecology}
Deep ecology is a somewhat recent branch of ecological philosophy (ecosophy) that considers humankind as an integral part of its environment. The philosophy emphasizes the interdependent value of human and non-human life as well as the importance of the ecosystem and natural processes. It provides a foundation for the environmental and green movements and has led to a new system of environmental ethics. Deep ecology's core principle is the claim that, like humanity, the living environment as a whole has the same right to live and flourish.
\end{sortEnvironment}

\begin{sortEnvironment}{IMF}
The International Monetary Fund (IMF) is an international organization of "189 countries working to foster global monetary cooperation, secure financial stability, facilitate international trade, promote high employment and sustainable economic growth, and reduce poverty around the world", formed in 1945 it came into existence with the goal of reconstructing the international payment system. It now plays a central role in the management of balance of payments difficulties and international financial crises. Countries contribute funds to a pool. 
\end{sortEnvironment}

\begin{sortEnvironment}{WB}
The World Bank is an international financial institution that provides loans to countries of the world for capital programs. The World Bank's stated official goal is the reduction of poverty. 
\end{sortEnvironment}

\begin{sortEnvironment}{WTO}
The World Trade Organization (WTO) is an intergovernmental organization that regulates international trade, officially signed in '94 by 123 nations. It is the largest international economic organization in the world. The WTO deals with regulation of trade in goods, services and intellectual property between participating countries.
\end{sortEnvironment}

\begin{sortEnvironment}{Kyoto Protocol}
The Kyoto Protocol is an international treaty which extends the 1992 United Nations Framework Convention on Climate Change (UNFCCC) that commits State Parties to reduce greenhouse gas emissions, based on the scientific consensus that (a) global warming is occurring and (b) it is extremely likely that human-made CO2 emissions have predominantly caused it. There are currently 192 parties (Canada withdrew). The Kyoto Protocol implemented the objective of the UNFCCC to fight global warming by reducing greenhouse gas concentrations in the atmosphere to "a level that would prevent dangerous anthropogenic interference with the climate system". The Protocol is based on the principle of common but differentiated responsibilities: it puts the obligation to reduce current emissions on developed countries on the basis that they are historically responsible for the current levels of greenhouse gases in the atmosphere.
\end{sortEnvironment}

\begin{sortEnvironment}{Rio Earth Summit}
The United Nations Conference on Environment and Development (UNCED), also known as the Rio de Janeiro Earth Summit was a major United Nations conference in 1992. In 2012, the United Nations Conference on Sustainable Development was also held in Rio, and is also commonly called Rio+20. 172 governments participated. Some 2,400 representatives of non-governmental organizations (NGOs) attended. An important achievement of the summit was an agreement on the Climate Change Convention which in turn led to the Kyoto Protocol and the Paris Agreement.
\end{sortEnvironment}

\begin{sortEnvironment}{Technology transfer}
Technology transfer is the process of transferring technology from the places and ingroups of its origination to wider distribution among more people and places.
\end{sortEnvironment}

\begin{sortEnvironment}{Biodiversity}
Biodiversity generally refers to the variety and variability of life on Earth. According to the United Nations Environment Programme, biodiversity typically measures variation at the genetic, the species, and the ecosystem level.
\end{sortEnvironment}

\begin{sortEnvironment}{UNCED}
United Nations Conference on Environment and Development (UNCED). Since 1990, the international community has convened 12 major conferences which have committed Governments to address urgently some of the most pressing problems facing the world today. Taken together, these high profile meetings have achieved a global consensus on the priorities for a new development agenda for the 1990s and beyond.
\end{sortEnvironment}

\begin{sortEnvironment}{UN}
The United Nations (UN) is an intergovernmental organization tasked to promote international co-operation and to create and maintain international order.
\end{sortEnvironment}

\begin{sortEnvironment}{WEF}
The World Economic Forum (WEF) is a Swiss nonprofit foundation. Recognized by the Swiss authorities as an international body, its mission is cited as "committed to improving the state of the world by engaging business, political, academic, and other leaders of society to shape global, regional, and industry agendas".
\end{sortEnvironment}

\begin{sortEnvironment}{WSF}
The World Social Forum is an annual meeting of civil society organizations which offers a self-conscious effort to develop an alternative future through the championing of counter-hegemonic globalization. The World Social Forum can be considered a visible manifestation of global civil society, bringing together non governmental organizations, advocacy campaigns, and formal and informal social movements seeking international solidarity. It tends to meet in January at the same time as its "great capitalist rival", the World Economic Forum's Annual Meeting Switzerland. 
\end{sortEnvironment}

\begin{sortEnvironment}{Urbanization}
Urbanization refers to the population shift from rural to urban areas.
\end{sortEnvironment}

\begin{sortEnvironment}{Civil society}
Civil society is the "aggregate of non-governmental organizations and institutions that manifest interests and will of citizens".
\end{sortEnvironment}