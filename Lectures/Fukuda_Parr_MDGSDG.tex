\chapter{Fukuda-Parr - From the MDGs to the SDGs}
\textit{26-10-2017 \\
Jan Hidde Lavell} 

\section*{Abstract}
The Sustainable Development Goals (SDGs) differ from the MDGs in
purpose, concept, and politics. This article focuses on the gender agenda in the SDGs as a reflection on the shifts from the MDGs to the SDGs. It argues that the SDGs address several of the key shortcomings of the MDGs and incorporate a broader and more transformative agenda that more adequately reflects the complex challenges of the 21st century, and the need for structural reforms in the global economy. The SDGs also reverse the MDG approach to global goal setting and the misplaced belief in the virtues of simplicity, concreteness, and quantification. While the SDGs promise the potential for a more transformative agenda, implementation will depend on continued advocacy on each of the targets to hold authorities to account.

\section*{SDGs differ from the MDGs in purpose, concept, and politics}
\begin{itemize}
\item First, the MDGs were a North-South aid agenda. The goals and targets were mostly relevant for developing countries only;
\item Second, the MDGs focused on poverty – understood as meeting basic needs – and its alleviation;
\item Third, the MDGs were drafted by technocrats who undertook limited consultations with other sources of knowledge and expertise, a process widely acknowledged as a major weakness.
\end{itemize}

\section*{Shortcomings of the MDGs}
Although the MDGs have been widely touted as a ‘success’, they have also been widely criticized. From the beginning, there was lukewarm reception by governments which suspected that they would become another source of aid conditionality. Civil society groups protested the omission of inequality, weak goals on global ‘partnership’ that lacked quantitative targets, the lack of ambition in the targets, and many omissions such as women’s reproductive health issues, governance, conflicts, economic growth and employment, and many other important objectives.\\
\\
In sum, stakeholders with a wide range of perspectives were deeply frustrated by the MDGs that came to dominate international development discourses. The targets were not ‘in synch’ with their agendas and vision, and disconnected from current national and global policy debates. They were concerned by the narrow breadth that was insufficiently ‘transformative’ to meet the challenges of development in the 21st century – that required a change, of course, not just continuation of business as usual, but shifts in institutions and economic models. The negotiations for the post-2015 agenda thus unleashed massive mobilization to correct the shortcomings of the MDGs. 

\section*{A broader and more transformative agenda}
In stark contrast to the technocratic process of elaborating the MDGs discussed in the last section, the formulation of the SDGs was consciously set up as a process of political negotiations amongst states. The formulation of the SDGs was a process of intense diplomatic negotiations and open multi-stakeholder debates (three years). \\
\\
Compared to the MDGs, the agenda of the SDGs is broader – with respect to gender as well as overall – and potentially more transformative. The SDGs do address many more aspects of these complex realities of women’s lives, and therefore represent a considerable advance on the MDGs, reflecting the participative and broader consultation process that led to them.

\section*{Potential pittfalls in implementation: selectivity, simplifications and national adaptation}

While the SDGs offer a broader agenda that has potential for course correction than the MDGs, will they make a difference? There is a risk that the most transformative goals and targets would be neglected in implementation through selectivity, simplification, and national adaptation. With 17 goals and 169 targets, which handful will receive policy attention, and mobilize effort and resources? Selectivity could lead to neglect of goals and targets that would address structural issues. It is widely believed that the MDGs mobilized action, yet not all goals and targets were the same. Some such as employment and hunger were poor cousins until the 2008 financial crisis and recession hit. Will SDG 10, to reduce inequality within and between countries, or Target 5.a, to ensure legal right of women to land ownership, receive attention? \\
\\
The carefully negotiated language of the 17-goal agenda, emphasizing intangible qualitative
objectives of equitable and sustainable development, has led to a complex language. The temptation would be to simplify this language, and strip away the important qualifiers. Already, a private initiative to publicize the SDGs – Global Goals – has simplified them, shortening the titles and reinterpreting them in the process. \\
\\ 
Another risk is the process of national adaptation. This reduces the political pressure on national governments to address the political causes of poverty and inequality. It can then be an invitation to water down the ambition of the SDGs. Implementation of the inequality goal is particularly challenging, as it is one of the few goals that requires a major change in course from the trends of the last decade. \\
\\
The SDGs are a politically negotiated consensus that has no enforcement mechanism built in. The onus falls on civil society groups to leverage the SDGs as course correction by putting pressure on governments and other powerful actors to account for the commitments made.
The MDGs had surprising effects. They were more effective than anyone expected in gaining traction as a dominant discourse of development. Their effects were not all benign. The SDGs have reversed the misplaced trust in simplicity as a virtue of global goal setting. It is hard to predict what consequences they will have. 
\clearpage