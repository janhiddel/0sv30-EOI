\chapter{Week 4 - Emerging Economies: China}
\textit{25-09-2017 \\
Johanna Höffken}

\section{What are the emerging economies and what effect do they have on the world?}
Emerging economies are countries that are restructuring their economies along market lines, entering globalized markets through opening up trade and tech transfers. For example, the BRICS. They made a transition from a developing to a developed country. However, their development will affect the intensity of natural resource use, pollution emission, consumption patterns and waste. 
\begin{itemize}
\item \textsc{Brazil
\item Russia
\item India
\item Costa-Rica
\item Sout-Africa}
\end{itemize}

\section{What happened to China between 1949 and 1978?}
Under Mao Zedong China made a great leap forward between 1958 and 1978. The focus lay mainly on the steel and grain production and environmental concerns were not on board. This had some major environmental consequences, such as deforestation, pollution and waste. Zedong also promoted the population growth: "more people can produce more". Under Mao's legacy there was social instability and political fragility. Political dissent was not tolerated and human rights like freedom of press were violated. 

\section{What happened to China between 1978 and 1992?}
Under Deng Xiaoping, China had a rapid industrialization and urbanization between 1978 and 1992. Xiaoping reformed the market with non-state-owned enterprises, liberalized foreign trade and investment regulations and relaxed the state price controls.  

\section{What are the principles of the diversifying development model?}
\begin{itemize}
\item Countries are in different development stages and therefore there is no universal model for SD;
\item Each nation needs to choose a suitable SD path that fits its context;
\item Common but differentiated responsibilities. 
\end{itemize}
 

\section{What are the key principles of China's SD model?}
\begin{enumerate}
\item It is underpinned by the Scientific Outlook on Development; 
\item It aims to promote social harmony and progress;
\item The model has economic growth as its top priority;
\item It is driven by the application of scientific and technological innovations;
\item It requires continued reform and the opening-up of the economy, 
\end{enumerate} 

China focuses on growth, despite the pledge for SD. Hu Jintao and later Xi Jinpin prosed the concept of green economic growth. The Scientific Outlook on Development (SOOD) criticizes the growth-at-all-costs economic strategy by Zedong and Xiaoping. 

\section{How did China put SD into action?}
China put SD into action with Five Year Plans (FYPs). China is a centrally planned economy and the FYP sketches legislation and economic, social and environmental policies and directions. Currently, China is in the 13th FYP (2016-2020) which accounts for innovation, coordination, green development and opening up and sharing. It embraces sustainable economic growth with emphasis on environment protection and inclusive development. \\
\\
China also works towards SD with command and control policies (e.g. limiting plastic bags) and price and market incencitives (eco-compensation). Promoting SD via population control also took place, with certain negative consequences like skewed sex rations as a result. \\
\\
China's uneven regional development led to a much richer western part of China in comparison to the other parts of China. Rich regions consume high-value goods produced with low-cost in a pollution intensive way in the poorer regions. 

\section{What are important environmental issues in China?}
\begin{itemize}
\item Structural pollution in the cities
\item Acid rain formation
\item Water scarcity for a great part of China's population
\item The conservation of nature
\end{itemize}
China's environmental problems have become world's problems. In 2012 the China's per capita footprint was two times as big as its available biocapacity. Integration with international environmental regimes is crucial: UN and WTO. China's environmental problems are trans-boundary. This pattern mirrors development in the industrialized world, which brought environmental destruction to the global south. 

\section{What is environmental leapfrogging?}
Environmental leapfrogging is the skipping of pollution intensive stages of development. Consequently, vintage technologies will be leapfrogged bu avoiding investments in older technologies. There are thee ways to catch up with industrialized countries:
\begin{enumerate}
\item Path creating catching up;
\item Path-skipping catching up;
\item Path-follow catching up. 
\end{enumerate}

\clearpage


