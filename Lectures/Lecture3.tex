\chapter{Week 3 - High consumption societies: the promotion of SD in the EU}
\textit{18-09-2017 \\
Johanna Höffken}

\section{How did the view of sustainable development arose in the EU?}
The European Union is an economic and political union of 28 member states. Since the end of WWII Western Europe had unprecedented levels of stability and prosperity. The continued economic growth fuels high consumption lifestyles and demand for products and services. However, modern European lifestyles are unsustainable. The EU draws down the resources of the other regions of the globe to meet it demands. Therefore, it is important that the EU promotes SD, for the well-being of the ecosystems and also for intra- and inter-generational equity and ensuring the world's health. 

\section{How did the EU began their way towards the preservation of the environment?}
EU initiatives for protecting environment began in 1972, a time with rising concerns (domestic and international) about environment. It was however not until Single European Act (SEA) in 1986 that the EU’s role in environmental protection was formally recognized. The EU feared that strengthening of environmental legislation by member states would hamper trade. 

\section{Which treaties were held in the EU?}
Treaty mandates were the gradual shift from environmental protection to sustainable development. 
\begin{itemize}
\item Treaty of European Union (Maastricht, 1992): set the objective for sustainable growth while respecting environment.
\item Treaty of Amsterdam (1997): call for balanced and SD for economic activities.
\item Treaty of Nice (2002) confirmed this.
\item Lisbon Treaty (2007): pursue SD within and beyond EU’s borders. Also the adoption of international measures to protect environment, combat climate change as objective of EU environmental policy. The Lisbon treaty recognizes the leading role of the EU  in addressing climate change. 
\end{itemize}

\section{What are the seven Environmental Action Programs (EAPs)?}
The EAPS guided the development of the EU's environmental policy since the 70s. 
\begin{itemize}
\item \textsc{1st} EAP (1974-1976): economic growth no end in itself
\item \textsc{2nd} EAP (1977-1981): limits of growth stemming from natural resource availability
\item \textsc{3rd} EAP (1982-1986): forged links between environmental policy and the community’s industrial strategy, arguing that environmental protection could stimulate innovation. Since 3rd EAP environmental protection is seen as contributing to EU’s competitiveness. This view replaced “limits to growth” argument in favor of a belief in continued economic growth based on efficiency in resource use. 
\item \textsc{4th} EAP (1987-1992): promotion of ecological modernization, especially twin focus on efficiency and technological innovation. 
\item \textsc{5th} EAP (1993-200): towards promoting SD: the first explicit pledge for promoting SD in the EU. The 5th EAP is EU’s response to to obligations incurred in Rio but also a moral obligation with the acceptance of common, but differentiated responsibilities. However, there were also critiques on the 5th EAP like the limited progress made, the lack of clarity and the non-binding character of the EAP. 
\item \textsc{6th} EAP: "Our future, Our choice" (2001-2010): more strategic and targeted approach. The 6th EAP had detailed measures in seven strategies and it was a participatory process when developing these. It was however weak in policy coherence and there was a lack of concrete targets. However, it kept environment on the EU policy agenda. 
\item \textsc{7th} EAP "Living well, within the limits of our planet" (2013-2020): helps to lend coherence to other EU environmental objectives and targets specified elsewhere and prioritizing growth. The 28 member states have different political and legal frameworks and entry of Lisbon Treaty. Many of the new member states prioritize growth and have limited interest in environment protection. The 7th EAP support other EU policy priorities such as economic growth and energy efficiency. 
\end{itemize}

\section{What is ecological modernization?}
Ecological modernization is an optimistic school of thought in the social sciences that argues that the economy benefits from moves towards environmentalism. It is an analytical approach as well as a policy strategy. 

\section{How did the EU put SD into practice?}
\begin{itemize}
\item With directives, the key legislative form of EU. Directives set end results, the way to achieve them is open to the member states.
\item With Environmental Policy Integration (EPI): the key challenge of integrating environmental concerns into other policies. EPI adresses the needs to enhance coherence between sectoral, economic, and environmental policies and between them and SD policies. 
\end{itemize}

\section{What were the critiques on the EU policy of promoting SD?}
\begin{itemize}
\item Lacks clear guidelines
\item Implementation issues
\item Actions addressing climate change shifted EU away from SD
\item EU prioritizes EU concerns
\item Emphasis of economic growth
\end{itemize}

\section{What is the alternative to the economic growth paradigm according to Van den Bergh (2011)?}
The theory of "degrowth" with five interpretations. The degrowth concept is ambigious. Agrowth is less ambigious and therefore considered as better. Agrowth is being neutral or indifferent to economic growth. 
\begin{enumerate}
\item GDP degrowth: striving for negative GDP growth or a reduction in GDP
\item Consumption degrowth: striving for a reduction in the amount of consumption. 
\item Work-time degrowth: shorter working weeks and more holidays, earlier retirement that will result in less production, lower wages and less consumption. 
\item Radical degrowth: radical change in values, ethics, preferences, financial systems, markets, work, profit-making and ownership. 
\item Physical degrowth: reduction of the physical size of the economy, resource use and emmissions: keeping economy within environmental limits
\end{enumerate}

\clearpage